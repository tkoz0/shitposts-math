\documentclass{article}[12pt]

\usepackage{parskip}
\usepackage{mathtools}
\usepackage{amsfonts}
\usepackage{mathrsfs}

\begin{document}

\section{Computing $4 \times 6$}

Is it 46? 64? 4444? 666666?

There are several ways to define multiplication. In the following sections, we
will explore them and analyze the results. It will be helpful to note that for
an integer $a>0$, the number of digits it has is $1+\lfloor\log_{10}(a)\rfloor$.

\subsection{Method 1}

Let $a>0$ and $b>0$ be integers. Then define concatenation multiplication as:
$$ a \times b = 10^{1 + \lfloor \log_{10}(b) \rfloor} \times a + b $$

Then $4 \times 6 = 46$. Note that by the commutative property of multiplication,
$4 \times 6 = 6 \times 4$. Using the multiplication we defined, $6 \times 4 =
64$. Then by the transitive property of equality, $46 = 64$ so we can conclude
that math is broken.

\subsection{Method 2}

Let $a>0$ and $b>0$ be integers. Define repeated digit multiplication as:
$$ a \times b = \sum_{i=0}^{a-1} b \times 10^{i\times(1+\lfloor\log_{10}(b)\rfloor)} $$

Similarly we compute $4 \times 6 = 6666$. And my using the commutative property
of multiplication, we get $6 \times 4 = 444444$. Finally, we use the transitive
property of equality like before and get $6666=444444$. Math is also broken.

\subsection{Conclusion}

There is no correct way to define multiplication of 2 integers because they all
break math.

\end{document}
